\appendix

\section{Mathematical Proofs}

This appendix provides rigorous mathematical proofs for the key theoretical claims made in the main text.

\subsection{Proof of Division Algebra Property}

\textbf{Theorem A.1:} The biquaternion algebra $\mathcal{B} = \mathcal{Q}_1 \otimes_{\mathbb{F}} \mathcal{Q}_2$ with $\mathcal{Q}_1 = \left(\frac{-1,-1}{\mathbb{F}}\right)$ and $\mathcal{Q}_2 = \left(\frac{\gamma,-1}{\mathbb{F}}\right)$ forms a division algebra for $\gamma = -i$.

\textbf{Proof:}
We need to show that every non-zero element in $\mathcal{B}$ has a multiplicative inverse.

\begin{enumerate}
    \item \textbf{Structure of $\mathcal{B}$:} The biquaternion algebra has dimension 16 over $\mathbb{F} = \mathbb{Q}(i)$. Any element $x \in \mathcal{B}$ can be written as:
    \begin{equation}
    x = q_1 \otimes 1 + q_2 \otimes \mathbf{J}
    \end{equation}
    where $q_1, q_2 \in \mathcal{Q}_1$ and $\mathbf{J}^2 = \gamma$.
    
    \item \textbf{Norm Form:} Define the reduced norm:
    \begin{equation}
    N(x) = \det(\mathbf{X}(x))
    \end{equation}
    where $\mathbf{X}(x)$ is the left regular representation matrix.
    
    \item \textbf{Non-vanishing Property:} For $\gamma = -i$, we verify that $N(x) \neq 0$ for all $x \neq 0$.
    
    The determinant of the left regular representation is:
    \begin{equation}
    \det(\mathbf{X}) = \det\begin{pmatrix}
    \psi(q_1) & \gamma \psi(q_2^{\sigma}) \\
    \psi(q_2) & \psi(q_1^{\sigma})
    \end{pmatrix}
    \end{equation}
    
    Using the block matrix determinant formula:
    \begin{equation}
    \det(\mathbf{X}) = \det(\psi(q_1))\det(\psi(q_1^{\sigma})) - \gamma \det(\psi(q_2))\det(\psi(q_2^{\sigma}))
    \end{equation}
    
    \item \textbf{Quaternion Norms:} For $q = x_0 + x_1\mathbf{i} + x_2\mathbf{j} + x_3\mathbf{k} \in \mathcal{Q}_1$:
    \begin{equation}
    \det(\psi(q)) = |x_0|^2 + |x_1|^2 + |x_2|^2 + |x_3|^2 = N_{\mathcal{Q}_1}(q)
    \end{equation}
    
    \item \textbf{Division Property:} For $\gamma = -i$, the norm becomes:
    \begin{equation}
    N(x) = N_{\mathcal{Q}_1}(q_1)^2 + |i| \cdot N_{\mathcal{Q}_1}(q_2)^2 = N_{\mathcal{Q}_1}(q_1)^2 + N_{\mathcal{Q}_1}(q_2)^2
    \end{equation}
    
    This is zero if and only if both $q_1 = 0$ and $q_2 = 0$, i.e., $x = 0$.
    
    \item \textbf{Inverse Construction:} For $x \neq 0$, the inverse is:
    \begin{equation}
    x^{-1} = \frac{\bar{x}}{N(x)}
    \end{equation}
    where $\bar{x}$ is the conjugate in $\mathcal{B}$.
\end{enumerate}

Therefore, $\mathcal{B}$ is a division algebra for $\gamma = -i$. $\square$

\subsection{Derivation of Minimum Determinant Formula}

\textbf{Theorem A.2:} The minimum determinant of the difference between distinct codewords is given by:
\begin{equation}
\delta_{min} = \min_{\substack{\mathbf{X}_1, \mathbf{X}_2 \in \mathcal{C} \\ \mathbf{X}_1 \neq \mathbf{X}_2}} |\det(\mathbf{X}_1 - \mathbf{X}_2)|
\end{equation}

\textbf{Derivation:}

\begin{enumerate}
    \item \textbf{Codeword Difference Structure:} For two distinct codewords $\mathbf{X}_1$ and $\mathbf{X}_2$ generated from symbol vectors $\mathbf{s}_1$ and $\mathbf{s}_2$:
    \begin{equation}
    \Delta\mathbf{X} = \mathbf{X}_1 - \mathbf{X}_2 = \mathbf{X}(\Delta q_1, \Delta q_2)
    \end{equation}
    where $\Delta q_i$ are the quaternion differences.
    
    \item \textbf{Determinant Expansion:} Using the block structure:
    \begin{equation}
    \det(\Delta\mathbf{X}) = \det\begin{pmatrix}
    \psi(\Delta q_1) & \gamma \psi(\Delta q_2^{\sigma}) \\
    \psi(\Delta q_2) & \psi(\Delta q_1^{\sigma})
    \end{pmatrix}
    \end{equation}
    
    \item \textbf{Schur Complement:} Assuming $\psi(\Delta q_1^{\sigma})$ is invertible:
    \begin{align}
    \det(\Delta\mathbf{X}) &= \det(\psi(\Delta q_1^{\sigma})) \times \\
    &\quad \det(\psi(\Delta q_1) - \gamma \psi(\Delta q_2^{\sigma})\psi(\Delta q_1^{\sigma})^{-1}\psi(\Delta q_2))
    \end{align}
    
    \item \textbf{Simplified Form:} For QPSK symbols with unit energy:
    \begin{equation}
    |\det(\Delta\mathbf{X})|^2 = |N_{\mathcal{Q}_1}(\Delta q_1)|^2 + |\gamma|^2 |N_{\mathcal{Q}_1}(\Delta q_2)|^2
    \end{equation}
    
    \item \textbf{Minimum Over Symbol Errors:} The minimum occurs for single-symbol errors:
    \begin{equation}
    \delta_{min}^2 = \min\{2, 2|\gamma|^2\} = 2\min\{1, |\gamma|^2\}
    \end{equation}
    
    For $\gamma = -i$, we have $|\gamma| = 1$, thus $\delta_{min}^2 = 2$.
\end{enumerate}

The coding gain is then:
\begin{equation}
\zeta = \delta_{min}^{1/2} = \sqrt[4]{2} \approx 1.189
\end{equation}
$\square$

\subsection{Proof of Full Diversity Achievement}

\textbf{Theorem A.3:} The biquaternion STBC achieves full transmit diversity order $d = 4$ for all non-zero codeword differences.

\textbf{Proof:}

\begin{enumerate}
    \item \textbf{Diversity Order Definition:} The diversity order is:
    \begin{equation}
    d = \min_{\Delta\mathbf{X} \neq 0} \text{rank}(\Delta\mathbf{X})
    \end{equation}
    
    \item \textbf{Rank-Determinant Relationship:} For a $4 \times 4$ matrix:
    \begin{equation}
    \text{rank}(\Delta\mathbf{X}) = 4 \iff \det(\Delta\mathbf{X}) \neq 0
    \end{equation}
    
    \item \textbf{Division Algebra Property:} From Theorem A.1, $\mathcal{B}$ is a division algebra, which means:
    \begin{equation}
    x \neq 0 \Rightarrow N(x) = \det(\mathbf{X}(x)) \neq 0
    \end{equation}
    
    \item \textbf{Codeword Differences:} For any two distinct codewords $\mathbf{X}_1, \mathbf{X}_2$:
    \begin{equation}
    \mathbf{X}_1 \neq \mathbf{X}_2 \Rightarrow \Delta\mathbf{X} = \mathbf{X}_1 - \mathbf{X}_2 \neq 0
    \end{equation}
    
    Since $\Delta\mathbf{X}$ corresponds to a non-zero element in $\mathcal{B}$:
    \begin{equation}
    \det(\Delta\mathbf{X}) \neq 0
    \end{equation}
    
    \item \textbf{Full Rank:} Therefore:
    \begin{equation}
    \text{rank}(\Delta\mathbf{X}) = 4 \quad \forall \Delta\mathbf{X} \neq 0
    \end{equation}
    
    \item \textbf{Diversity in Fading:} The pairwise error probability in Rayleigh fading:
    \begin{equation}
    P(\mathbf{X}_1 \to \mathbf{X}_2) \leq \left(\frac{1}{4\text{SNR}}\right)^{d \cdot N_r}
    \end{equation}
    
    With $d = 4$ and $N_r$ receive antennas, the diversity order is $4N_r$.
\end{enumerate}

Thus, the biquaternion STBC achieves full transmit diversity. $\square$

\subsection{Convergence Proof for MMSE Iterative Detector}

\textbf{Theorem A.4:} The iterative MMSE detector with soft interference cancellation converges to a fixed point within $K$ iterations.

\textbf{Proof:}

Consider an iterative MMSE detector with soft interference cancellation:

\begin{enumerate}
    \item \textbf{Iteration Update:} At iteration $k$, the estimate for symbol $i$ is:
    \begin{equation}
    \hat{x}_i^{(k+1)} = \mathbf{w}_i^H \left(\mathbf{y} - \sum_{j \neq i} \mathbf{h}_j \hat{x}_j^{(k)}\right)
    \end{equation}
    where $\mathbf{w}_i$ is the MMSE filter for symbol $i$.
    
    \item \textbf{MMSE Filter:} The filter is:
    \begin{equation}
    \mathbf{w}_i = \left(\sum_{j \neq i} \mathbf{h}_j \mathbf{h}_j^H + \sigma^2 \mathbf{I}\right)^{-1} \mathbf{h}_i
    \end{equation}
    
    \item \textbf{Fixed Point Equation:} At convergence, $\hat{x}_i^{(k+1)} = \hat{x}_i^{(k)} = \hat{x}_i^*$:
    \begin{equation}
    \hat{x}_i^* = \mathbf{w}_i^H \left(\mathbf{y} - \sum_{j \neq i} \mathbf{h}_j \hat{x}_j^*\right)
    \end{equation}
    
    \item \textbf{Contraction Mapping:} Define the update function:
    \begin{equation}
    F(\mathbf{x}) = \begin{pmatrix}
    f_1(\mathbf{x}) \\
    \vdots \\
    f_N(\mathbf{x})
    \end{pmatrix}
    \end{equation}
    where $f_i(\mathbf{x}) = \mathbf{w}_i^H(\mathbf{y} - \sum_{j \neq i} \mathbf{h}_j x_j)$.
    
    \item \textbf{Lipschitz Continuity:} The Jacobian of $F$ is:
    \begin{equation}
    J_{ij} = \frac{\partial f_i}{\partial x_j} = -\mathbf{w}_i^H \mathbf{h}_j \quad (i \neq j)
    \end{equation}
    
    \item \textbf{Spectral Radius:} The spectral radius of $J$ satisfies:
    \begin{equation}
    \rho(J) < 1 \quad \text{when} \quad \text{SNR} > \text{SNR}_{threshold}
    \end{equation}
    
    For our system with $\gamma = -i$ and typical channel conditions:
    \begin{equation}
    \text{SNR}_{threshold} \approx -5 \text{ dB}
    \end{equation}
    
    \item \textbf{Convergence Rate:} By the Banach fixed-point theorem:
    \begin{equation}
    \|\mathbf{x}^{(k)} - \mathbf{x}^*\| \leq \rho(J)^k \|\mathbf{x}^{(0)} - \mathbf{x}^*\|
    \end{equation}
    
    \item \textbf{Iteration Bound:} For convergence within tolerance $\epsilon$:
    \begin{equation}
    K = \left\lceil \frac{\log(\epsilon/\|\mathbf{x}^{(0)} - \mathbf{x}^*\|)}{\log(\rho(J))} \right\rceil
    \end{equation}
    
    Typically, $K \leq 5$ iterations suffice for $\epsilon = 10^{-3}$.
\end{enumerate}

Therefore, the iterative MMSE detector converges to a unique fixed point. $\square$

\subsection{Additional Proof: Non-Vanishing Determinant Property}

\textbf{Theorem A.5:} The minimum determinant of the biquaternion STBC remains bounded away from zero as the constellation size increases.

\textbf{Proof:}

\begin{enumerate}
    \item \textbf{Constellation Scaling:} Consider $M$-QAM constellation with average energy $E_s$:
    \begin{equation}
    \mathcal{S}_M = \left\{s : s = a + jb, \; a,b \in \left\{\pm 1, \pm 3, \ldots, \pm(\sqrt{M}-1)\right\}\right\}
    \end{equation}
    
    \item \textbf{Normalized Constellation:} After normalization:
    \begin{equation}
    \tilde{\mathcal{S}}_M = \frac{1}{\sqrt{E_s}} \mathcal{S}_M
    \end{equation}
    
    \item \textbf{Minimum Symbol Distance:} The minimum distance is:
    \begin{equation}
    d_{min} = \frac{2}{\sqrt{E_s}} = \sqrt{\frac{6}{M-1}}
    \end{equation}
    
    \item \textbf{Codeword Determinant:} For single-symbol error:
    \begin{equation}
    |\det(\Delta\mathbf{X})| \geq C \cdot d_{min}^4
    \end{equation}
    where $C$ depends on the algebraic structure but not on $M$.
    
    \item \textbf{Lower Bound:} For our biquaternion construction:
    \begin{equation}
    |\det(\Delta\mathbf{X})| \geq \frac{C'}{(M-1)^2}
    \end{equation}
    
    \item \textbf{NVD Property:} As $M \to \infty$:
    \begin{equation}
    \liminf_{M \to \infty} M^2 \cdot \min_{\Delta\mathbf{X} \neq 0} |\det(\Delta\mathbf{X})| = C' > 0
    \end{equation}
    
    This confirms the non-vanishing determinant property.
\end{enumerate}

The NVD property ensures that the diversity advantage is maintained even for large constellation sizes. $\square$
