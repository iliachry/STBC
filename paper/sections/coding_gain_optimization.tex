\section{Coding Gain Optimization}
In this section, we delve into the optimization of the coding gain for the proposed STBCs. 
While the division algebra structure guarantees full diversity, the actual error performance in finite-SNR regimes is largely determined by the coding gain. 
We first explain the theoretical foundation of coding gain and then present our systematic optimization approach. 
Importantly, our analysis reveals that optimization benefits are critically dependent on the detection algorithm employed at the receiver.

\subsection{The Coding Gain Criterion}
The error performance of an STBC is fundamentally tied to the pairwise error probability (PEP), which is the probability that the receiver mistakes one transmitted codeword $\mathbf{X}_1$ for another $\mathbf{X}_2$. 
Under the assumption of maximum-likelihood (ML) decoding in a Rayleigh fading channel with perfect CSI at the receiver, the PEP for a given channel realization $\mathbf{H}$ is upper-bounded by:
\begin{equation}
P(\mathbf{X}_1 \to \mathbf{X}_2 | \mathbf{H}) \leq \exp\left( -\frac{\|\mathbf{H}(\mathbf{X}_1 - \mathbf{X}_2)\|_F^2}{4N_0} \right),
\end{equation}
where $\|\cdot\|_F$ denotes the Frobenius norm and $N_0$ is the noise variance. Averaging over the channel statistics, the PEP behaves asymptotically as $\text{PEP} \sim \left( \frac{\rho}{4} \right)^{-N_r N_t} \cdot \zeta^{-N_r}$, where $\rho$ is the SNR and $\zeta$ is the coding gain.

The coding gain $\zeta$ is defined as:
\begin{equation}
\zeta = \min_{\Delta\mathbf{X} \neq \mathbf{0}} |\det(\Delta\mathbf{X})|^{2/N_t},
\end{equation}
where $\Delta\mathbf{X} = \mathbf{X}_1 - \mathbf{X}_2$ is the difference matrix between any two distinct codewords, and the minimum is taken over all possible non-zero differences. A larger $\zeta$ shifts the BER curve leftward, improving performance at all SNRs. The non-vanishing determinant (NVD) property, ensured by the division algebra structure, guarantees that $\zeta > 0$ and does not diminish as the constellation size increases. However, the NVD only provides a lower bound; our objective is to maximize $\zeta$ to achieve the best possible error rates.

Crucially, the practical impact of coding gain optimization depends significantly on the receiver's detection complexity. 
While theoretical analysis typically assumes optimal ML detection, real-world systems often employ suboptimal linear detection schemes such as minimum mean square error (MMSE) or zero-forcing (ZF) for computational efficiency.
Our investigation reveals that these suboptimal detectors amplify the visibility of coding gain differences, making parameter optimization increasingly valuable as detection complexity decreases.

\subsection{The Optimization Problem}
From the codeword matrix in \eqref{eq:stbc_matrix_general}, the determinant of a difference matrix $\Delta\mathbf{X}$ depends on the quaternion differences $\Delta q_1$, $\Delta q_2$, and the parameter $\gamma$. The structure of the biquaternion matrix ensures that $\gamma$ directly influences the minimum determinant through its appearance in the upper-right block, affecting the overall geometric properties of the code.

The optimization problem seeks the value of $\gamma$ that maximizes the minimum squared determinant:
\begin{equation} \label{eq:optimization}
\gamma_{opt} = \arg \max_{\gamma \in \mathbb{F}} \left( \min_{\substack{\Delta q_1, \Delta q_2 \in \mathcal{Q}_1 \\ (\Delta q_1, \Delta q_2) \neq (0,0)}} |\det(\mathbf{X}(\Delta q_1, \Delta q_2))|^2 \right).
\end{equation}

An analytical solution is intractable due to: 
(i) the discrete nature of symbol constellations (e.g., QPSK points are finite and non-uniformly distributed), and 
(ii) the high-dimensional search space over possible error events $\Delta s$, which determine $\Delta q_1$ and $\Delta q_2$ through the symbol mapping.

\subsection{Practical Optimization Approach}
To solve this optimization problem, we employ a systematic numerical approach that balances computational feasibility with thorough exploration of the parameter space.

\textbf{Grid-Based Search Strategy:} For a given constellation over number field $\mathbb{F} = \mathbb{Q}(i)$, we discretize the search space for $\gamma$ over a structured grid of candidate values. 
Rather than exhaustive search, we focus on regions around theoretically motivated values, such as those derived from golden ratio principles that have proven effective in related algebraic constructions.

\textbf{Error Event Enumeration:} For each candidate $\gamma$:
\begin{enumerate}
    \item Generate representative symbol error vectors $\Delta s$ based on typical constellation differences. 
    For QPSK rate-2 designs, this involves systematic sampling of the most likely error patterns rather than exhaustive enumeration.
    
    \item Map each $\Delta s$ to the corresponding quaternion differences $\Delta q_1, \Delta q_2$ using the established symbol mapping equations.
    
    \item Construct the difference matrix $\mathbf{X}(\Delta q_1, \Delta q_2)$ and compute $|\det(\mathbf{X})|^2$.
    
    \item Record the minimum value across all tested error events.
\end{enumerate}

\textbf{Selection Criterion:} The $\gamma$ value yielding the largest minimum determinant is selected as $\gamma_{opt}$. 
This optimization is performed offline during code design, making computational complexity less critical than accuracy.

\textbf{Practical Considerations:} Our optimization approach recognizes that different detection schemes reveal optimization benefits to varying degrees. 
While the same $\gamma_{opt}$ theoretically maximizes coding gain regardless of detector, the practical BER improvements scale with detection suboptimality. 
This suggests that optimization is most valuable for systems employing linear detection schemes, where computational constraints prevent optimal ML decoding.

The systematic nature of this approach ensures that the selected $\gamma$ provides meaningful coding gain improvements across various operating conditions, with benefits becoming increasingly pronounced as receiver complexity decreases from optimal ML to practical linear detection algorithms.
