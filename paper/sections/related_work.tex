\section{Related Work}

The field of space-time coding has evolved from simple orthogonal structures to complex algebraic designs. 
The Alamouti code provided an elegant full-rate, full-diversity solution for two antennas with simple linear decoding \cite{5}. 
However, it was soon proven that complex orthogonal designs for more than two antennas must sacrifice rate, with a maximum achievable rate of 3/4 for four antennas \cite{1}.

To overcome this rate limitation, research turned to non-orthogonal designs. 
The fundamental limits of the rate-diversity tradeoff were characterized in the seminal work by Zheng and Tse \cite{2}. 
This motivated the search for codes that could approach these limits. 
Division algebras over number fields were identified as a powerful tool for constructing STBCs that are full-rate and full-diversity \cite{3}. 
The most famous example is the \emph{Golden Code}, a 2x2 STBC built from a cyclic division algebra that achieves the optimal diversity-multiplexing tradeoff \cite{6}.

For larger 4x4 MIMO systems, research has explored more complex algebraic structures, including cyclic division algebras and biquaternion algebras \cite{7}. 
Biquaternion algebras, formed as the tensor product of two quaternion algebras, provide a rich structure for building 4x4 codes. 
Several works have proposed constructions based on this approach \cite{8,9}. 
However, as noted in comprehensive surveys on the topic \cite{10}, these constructions often contain degrees of freedom (such as the choice of a parameter like our $\gamma$) whose impact on the coding gain is not explicitly optimized. 
The focus has largely been on guaranteeing the NVD property and achieving full rate, while the crucial task of maximizing the minimum determinant is often left as an open problem.

Importantly, the practical evaluation of STBC optimization has largely focused on maximum likelihood (ML) detection, which provides optimal performance but requires high computational complexity \cite{11,12}. 
Recent work has shown that suboptimal linear detection schemes such as minimum mean square error (MMSE) and zero-forcing (ZF) are more practical for real-world implementations \cite{13,14}. 
However, the impact of detection complexity on the visibility of coding gain optimization has received limited attention in the literature. 
Our work directly addresses both gaps by providing systematic optimization methods and comprehensive analysis across different detection schemes.
