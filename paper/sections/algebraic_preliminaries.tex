\section{Algebraic Preliminaries}
To construct our tunable-rate STBCs, we leverage the structure of quaternion and biquaternion algebras, which are types of central simple algebras over number fields. 
These algebras provide a rich framework for embedding information symbols into matrices that ensure desirable properties like full diversity and high rates in MIMO systems. 
Below, we detail the key concepts underlying our construction.

\subsection{Quaternion Algebras}

A generalized quaternion algebra $\left(\frac{a, b}{\mathbb{F}}\right)$ over a field $\mathbb{F}$ (typically a number field with characteristic not equal to 2) is a four-dimensional central simple algebra over $\mathbb{F}$. It is "central" because its center is exactly $\mathbb{F}$, and "simple" because it has no non-trivial two-sided ideals. The algebra has a basis $\{1, \mathbf{i}, \mathbf{j}, \mathbf{k}\}$ with multiplication rules defined by:
\begin{equation}
\mathbf{i}^2 = a, \quad \mathbf{j}^2 = b, \quad \mathbf{ij} = -\mathbf{ji} = \mathbf{k},
\end{equation}
where $a, b \in \mathbb{F}^\times$ (the non-zero elements of $\mathbb{F}$). An arbitrary element $q \in \left(\frac{a, b}{\mathbb{F}}\right)$ can be expressed as $q = x_0 + x_1 \mathbf{i} + x_2 \mathbf{j} + x_3 \mathbf{k}$, with $x_0, x_1, x_2, x_3 \in \mathbb{F}$.

A well-known example is Hamilton's quaternion algebra $\mathbb{H} = \left(\frac{-1, -1}{\mathbb{R}}\right)$, where $\mathbf{i}^2 = \mathbf{j}^2 = -1$ and $\mathbf{k}^2 = -1$. Quaternion algebras are either division algebras (every non-zero element has a multiplicative inverse) or isomorphic to the matrix algebra $M_2(\mathbb{F})$. The division property is determined by the Hilbert symbol, which evaluates whether the quadratic form $a x^2 + b y^2 - z^2 = 0$ has non-trivial solutions in $\mathbb{F}$.

In the context of STBC design, quaternion algebras are useful because they can be represented as $2 \times 2$ matrices over $\mathbb{F}$, allowing information symbols to be mapped to matrix entries. This representation preserves the algebraic structure, enabling codes that achieve high rates while maintaining good distance properties \cite{3}.

\subsection{Biquaternion Division Algebras}

A biquaternion algebra $\mathcal{B}$ extends the quaternion concept by taking the tensor product of two quaternion algebras over $\mathbb{F}$:
\begin{equation}
\mathcal{B} = \mathcal{Q}_1 \otimes_{\mathbb{F}} \mathcal{Q}_2 = \left(\frac{a, b}{\mathbb{F}}\right) \otimes_{\mathbb{F}} \left(\frac{c, d}{\mathbb{F}}\right).
\end{equation}
This results in a 16-dimensional algebra over $\mathbb{F}$, as the tensor product combines the 4-dimensional bases of $\mathcal{Q}_1$ and $\mathcal{Q}_2$ to form 16 basis elements. For our construction, we specifically choose $\mathcal{Q}_1 = \left(\frac{-1, -1}{\mathbb{F}}\right)$ and $\mathcal{Q}_2 = \left(\frac{\gamma, -1}{\mathbb{F}}\right)$, where $\gamma$ is the key optimization parameter that governs the coding gain.

\textbf{Albert's theorem} on tensor products of quaternion algebras provides the theoretical foundation for our approach. It states that $\mathcal{B} = \mathcal{Q}_1 \otimes_{\mathbb{F}} \mathcal{Q}_2$ is a division algebra if and only if there is no common quadratic subfield between $\mathcal{Q}_1$ and $\mathcal{Q}_2$ \cite{3}. This condition can be verified using ramification conditions at the places of $\mathbb{F}$. In our specific construction with $\mathcal{Q}_1 = \left(\frac{-1, -1}{\mathbb{F}}\right)$ and $\mathcal{Q}_2 = \left(\frac{\gamma, -1}{\mathbb{F}}\right)$, the division property holds for appropriate choices of $\gamma$, ensuring the algebras ramify at different primes.

For STBC design, the division property is crucial because it ensures the non-vanishing determinant (NVD) property: the determinant of any non-zero codeword matrix is non-zero and bounded away from zero as the constellation size increases. This guarantees full transmit diversity of order $N_t = 4$, making the code robust to fading channels. Moreover, the 16-dimensional structure of $\mathcal{B}$ provides exactly the degrees of freedom needed for a $4 \times 4$ matrix codeword, enabling tunable-rate designs that can transmit either 4 symbols (rate $R=1$) or 8 symbols (rate $R=2$) over 4 time slots.

\subsection{Left Regular Representation and Matrix Embedding}

The connection between the abstract biquaternion algebra and practical STBC matrices is established through the \emph{left regular representation}. 
For an element $x \in \mathcal{B}$, which can be uniquely written as $x = q_1 + q_2 \mathbf{J}$ where $q_1, q_2 \in \mathcal{Q}_1$, the left regular representation yields a $4 \times 4$ complex matrix of the canonical form:
\begin{equation}
\mathbf{X}(q_1, q_2) = 
\begin{pmatrix}
\psi(q_1) & \gamma \psi(q_2^{\sigma}) \\
\psi(q_2) & \psi(q_1^{\sigma})
\end{pmatrix},
\end{equation}
where $\sigma$ is an involution on $\mathcal{Q}_1$ and $\psi(\cdot)$ maps quaternions to $2 \times 2$ complex matrices.

The parameter $\gamma$ appears explicitly in the upper-right block and directly affects the singular values of $\mathbf{X}$, influencing the code's minimum distance and thus its coding gain.
This provides the theoretical foundation for the systematic optimization approach developed in this work, where we search for values of $\gamma$ that maximize the minimum determinant and thus improve error performance across different detection schemes.
