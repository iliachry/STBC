\section{Simulation Results and Analysis}
We evaluate the performance of our proposed STBCs through extensive Monte Carlo simulations in a 4x4 MIMO system (\(N_t = N_r = 4\)) under quasi-static Rayleigh fading conditions. The simulations are conducted with \(10^5\) independent channel realizations per SNR point to ensure statistical reliability. We assume perfect channel state information (CSI) at the receiver and employ maximum likelihood (ML) decoding, approximated via sphere decoding for computational efficiency. The transmitted symbols are drawn from a QPSK constellation, normalized to unit energy per symbol.

We compare three schemes to highlight the benefits of our optimization and the tunable-rate framework. The parameters for these schemes are summarized in Table \ref{tab:params}. The "Optimized Code" uses the \(\gamma_{opt}\) found from our numerical search, the "Non-Optimized" uses a typical literature value for algebraic independence, and the "Robust Code" is the rate-1 design with \(q_2 = 0\).

\begin{table}[h]
\caption{Parameters of Simulated STBC Schemes}
\label{tab:params}
\centering
\begin{tabular}{|l|c|c|c|}
\hline
\textbf{Scheme} & \textbf{Rate (R)} & \textbf{Modulation} & \textbf{Parameter \(\gamma\)} \\
\hline
Optimized Code & 2 & QPSK & \(0.4 + 1.1i\) \\
Non-Optimized & 2 & QPSK & \(1 + i\) \\
Robust Code & 1 & QPSK & N/A (\(q_2=0\)) \\
\hline
\end{tabular}
\end{table}

Fig. \ref{fig:ber_plot} presents the BER performance as a function of SNR. The key observations are as follows:
\begin{itemize}
    \item \textbf{Diversity Slope Confirmation:} All three curves exhibit a slope of approximately \(-4\) at high SNR (e.g., beyond 20 dB), visually confirming that they achieve the full diversity order of 4. This aligns with the theoretical guarantee provided by the division algebra structure and the NVD property, as proven in our framework.
    
    \item \textbf{Coding Gain Advantage:} The optimized rate-2 code consistently outperforms the non-optimized version across all SNRs. At a target BER of \(10^{-5}\), the optimized code requires about 1.5 dB less SNR than the non-optimized one. This gain stems directly from the larger minimum determinant (\(\zeta\)) achieved by \(\gamma_{opt}\), which increases the effective distance between codewords and reduces error events. Importantly, this improvement comes at no additional cost in terms of transmit power, bandwidth, or receiver complexity.
    
    \item \textbf{Rate-Performance Tradeoff:} The rate-1 robust design offers the best BER performance, requiring roughly 3 dB less SNR than the optimized rate-2 code at \(10^{-5}\) BER. This illustrates the explicit tradeoff enabled by our framework: lower rates provide higher reliability and simpler decoding (due to the block-diagonal structure), while higher rates prioritize spectral efficiency in favorable channel conditions. This tunability is particularly valuable for adaptive systems.
    
    \item \textbf{Additional Insights:} At low SNRs (below 10 dB), the robust design's advantage is more pronounced due to its reduced number of symbols, which lowers the likelihood of detection errors. As SNR increases, the rate-2 designs converge to their diversity-limited performance, with the optimization providing a consistent edge. These results were obtained with 4 receive antennas; similar trends hold for fewer receivers, though with reduced diversity gain.
\end{itemize}

To further quantify the optimization's impact, we computed the empirical minimum determinant across 10,000 random codeword pairs for the rate-2 designs. The optimized code yields a minimum determinant 2.3 times larger than the non-optimized one, directly correlating with the observed BER shift.

\begin{figure}[!t]
\centering
\includegraphics[width=0.9\columnwidth]{ber_plot.eps} 
\caption{BER performance comparison of the optimized rate-2 code, a non-optimized version, and the rate-1 design in a 4x4 Rayleigh fading channel.}
\label{fig:ber_plot}
\end{figure}

These results validate the practical value of our framework and optimization method, demonstrating that algebraic parameter tuning can yield substantial gains in real-world MIMO systems.
