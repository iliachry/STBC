\section{Conclusion}
This paper has addressed the critical but often overlooked issue of coding gain optimization in the design of algebraic STBCs for 4x4 MIMO systems. We began by presenting a flexible framework for constructing tunable-rate STBCs from biquaternion division algebras, which provides a solid foundation for achieving both full diversity and high spectral efficiency. The core contribution of our work was the introduction of a systematic method to maximize the code's performance by optimizing a key algebraic parameter, \(\gamma\).

Our simulation results have conclusively demonstrated the practical value of this approach. By moving beyond arbitrary parameter choices and performing a structured numerical search, we identified an optimal \(\gamma = -i\) that yields significant performance gains across multiple detection schemes. Most notably, the impact of optimization varies dramatically with detector complexity: with ML detection, the optimized code achieves error-free transmission at 6 dB SNR, compared to standard code which achieves the same at 6 dB and poor code at 10 dB. For MMSE detection, the optimized code's advantage is more pronounced, with a 4x reduction in bit error rate at 10 dB SNR compared to standard code and 21x improvement over poor code. For computationally simplest ZF detection, we observed performance varying by over 50% depending on parameter selection. These improvements are achieved without additional implementation complexity, power, or bandwidth, effectively unlocking the full potential of the underlying algebraic structure.

Looking ahead, this optimized framework serves as a launchpad for two promising research directions. The first is the further development of the novel detectors we introduced: the ML-Enhanced ZF, Adaptive MMSE, and Hybrid detectors. Our results demonstrated that ML-ZF and Hybrid detectors achieve ML-level performance at substantially reduced complexity for well-conditioned channels, while the Adaptive MMSE detector provides robust performance across diverse channel conditions. The second direction is to develop adaptive detector selection based on real-time channel conditions. Such a system would dynamically choose between computationally efficient detectors like ZF-REG for favorable channels and more robust options like Hybrid detection for challenging conditions. This adaptive approach, combined with our optimized algebraic parameter selection, could yield significant energy efficiency improvements for next-generation wireless systems.
