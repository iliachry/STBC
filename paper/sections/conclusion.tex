\section{Conclusion}
This paper has addressed the critical but often overlooked issue of coding gain optimization in the design of algebraic STBCs for 4x4 MIMO systems. We began by presenting a flexible framework for constructing tunable-rate STBCs from biquaternion division algebras, which provides a solid foundation for achieving both full diversity and high spectral efficiency. The core contribution of our work was the introduction of a systematic method to maximize the code's performance by optimizing a key algebraic parameter, \(\gamma\).

Our extensive 100,000-trial Monte Carlo simulation results have conclusively demonstrated the practical value of this approach. By moving beyond arbitrary parameter choices and performing a structured numerical search, we identified an optimal \(\gamma = -i\) that yields significant performance gains across multiple detection schemes. Most notably, the impact of optimization varies dramatically with detector complexity: with ML detection, the optimized code achieves error-free transmission at 10 dB SNR within our simulation limits. Enhanced detectors provide excellent complexity-performance trade-offs, with Adaptive MMSE achieving near-ML performance (BER = $6.63 \times 10^{-4}$ at 10 dB) while requiring 44\% less computation time, and Hybrid detection maintaining excellent performance (BER = $9.98 \times 10^{-4}$ at 10 dB) with 43\% computational savings. For basic linear detectors, the differences are stark: MMSE achieves BER = $1.97 \times 10^{-3}$ while ZF shows significant degradation (BER = 0.127 at 10 dB), demonstrating the critical importance of proper algorithmic choice. These improvements are achieved without additional implementation complexity, power, or bandwidth, effectively unlocking the full potential of the underlying algebraic structure.

Looking ahead, this optimized framework serves as a launchpad for two promising research directions. The first is the further development of the enhanced detectors we introduced: the Adaptive ZF, Adaptive MMSE, and Hybrid detectors. Our results demonstrated that Adaptive MMSE and Hybrid detectors achieve near-ML performance at substantially reduced complexity, while the Adaptive ZF detector provides a good balance between complexity and performance for moderately challenging channel conditions (BER = $1.26 \times 10^{-2}$ at 10 dB, still 10$\times$ better than basic ZF). The second direction is to develop adaptive detector selection based on real-time channel conditions and computational constraints. Such a system would dynamically choose between computationally efficient detectors like ZF-REG for favorable channels and more robust options like Adaptive MMSE or Hybrid detection for challenging conditions. This adaptive approach, combined with our optimized algebraic parameter selection, could yield significant energy efficiency improvements for next-generation wireless systems.
