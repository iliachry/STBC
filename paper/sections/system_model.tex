\section{System Model}

We consider a point-to-point multiple-input multiple-output (MIMO) communication system with $N_t = 4$ transmit antennas and $N_r$ receive antennas, operating over a quasi-static flat-fading channel. 
The channel coefficients $h_{j,i}$ connecting the $i$-th transmit antenna to the $j$-th receive antenna are modeled as independent and identically distributed (i.i.d.) complex Gaussian random variables with zero mean and unit variance, representing Rayleigh fading.

The information symbols are drawn from a complex constellation (e.g., QPSK) and encoded into space-time block code (STBC) matrices $\mathbf{X} \in \mathbb{C}^{N_t \times T}$ of dimension $4 \times 4$, where $T = 4$ time slots correspond to one transmission block. 
Each entry $x_{i,t}$ represents the complex symbol transmitted from antenna $i$ at time slot $t$.

The received signal at antenna $j$ during time slot $t$ is given by:
\begin{equation}
y_{j,t} = \sum_{i=1}^{N_t} h_{j,i} x_{i,t} + n_{j,t}
\end{equation}
where $n_{j,t}$ is additive white Gaussian noise (AWGN) with zero mean and variance $N_0/2$ per real and imaginary component.

The complete transmission over $T$ time slots can be expressed in matrix form as:
\begin{equation} \label{eq:system_model_matrix}
\mathbf{Y} = \mathbf{H}\mathbf{X} + \mathbf{N}
\end{equation}
where $\mathbf{Y} \in \mathbb{C}^{N_r \times T}$ is the received signal matrix, $\mathbf{H} \in \mathbb{C}^{N_r \times N_t}$ is the channel matrix, and $\mathbf{N} \in \mathbb{C}^{N_r \times T}$ is the noise matrix with i.i.d. complex Gaussian entries.

We assume perfect channel state information (CSI) is available at the receiver, while the transmitter has no CSI. 
The receiver's objective is to detect the transmitted symbols from $\mathbf{Y}$ using various detection algorithms.

For detection purposes, the system can be reformulated in vectorized form as:
\begin{equation}
\mathbf{y} = \mathcal{H}\mathbf{s} + \mathbf{n}
\end{equation}
where $\mathbf{y} = \text{vec}(\mathbf{Y})$ is the vectorized received signal, $\mathbf{s}$ contains the transmitted information symbols, $\mathcal{H}$ is the equivalent channel matrix that incorporates both the channel matrix $\mathbf{H}$ and the STBC structure, and $\mathbf{n} = \text{vec}(\mathbf{N})$ is the vectorized noise.

This equivalent representation enables the application of various detection algorithms, including maximum likelihood (ML), minimum mean square error (MMSE), and zero-forcing (ZF) detection, whose performance characteristics with respect to coding gain optimization form a central focus of this work.
